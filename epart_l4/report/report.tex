% Preamp
\documentclass[
  a4paper,            % DIN A4
  DIV=10,             % Schriftgröße und Satzspiegel
  oneside,            % einseitiger Druck
  BCOR=5mm,           % Bindungskorrektur
  parskip=half,       % Halber Abstand zwischen Absätzen
  numbers=noenddot,   % Kein Punkt hinter Kapitelnummern
  bibtotoc,           % Literaturverzeichnis im Inhaltsverzeichnis
  listof=totoc,        % Abbildungs- und Tabellenverzeichnis im Inhaltsverzeichnis
  article
]{scrreprt}
\usepackage{./convenience}

\usepackage{amsmath}
\usepackage{graphicx}
\usepackage{hyperref}
\usepackage{listings}
\usepackage{xcolor}

\counterwithout{figure}{section}
\renewcommand{\thesection}{\arabic{section}}

\RequirePackage{graphicx}           % Include graphics to document
\usepackage{float}                  % Better image placement
\RequirePackage{subfig}             % Multiple Pictures next to each other
\RequirePackage[top=1cm, right=1.5cm, bottom=2cm, left=1.5cm]{geometry}           % Change the page margins

% orphans and widows are typographically bad
\clubpenalty=10000
\widowpenalty=10000

% Document

\title{EPART Neural Networks Lab Report}
\author{Rustenis Tolpeznikas}
\date{\today}

\begin{document}

\maketitle
\newpage

\section{Implementation}
\small{Note: Since in this solution contains no rejection rates, while also having the OK rate being $1 - error\_rate$, I will only include error rates in the tables.}
\\
The initial implementation part went quite smoothly, as I just followed the instructions of the video uploaded at \textit{MS Teams}.
With 100 hidden layer neurons, 50 epochs, leaning rate of 0.001 and $tanh()$ activation function, I obtained a reference solution, which will be used as the benchmark solution:
\includenamedimage[width=0.6\textwidth]{ref.png}{Reference Solution}
\begin{table}[h!]
\centering
\begin{tabular}{|c|c|}
\hline
\textbf{Train Error} & \textbf{Test Error} \\
\hline
0.0721 & 0.1194 \\
\hline
\end{tabular}
\caption*{Reference Solution Error Rates}
\end{table}

\section{Testing Different Hyperparameters}
Next, I tested different Hyperparameters (Hidden Layer Neuron amount as well as the Learning Rate) to see if I can achieve better results.
\subsection{Hidden Layer Neurons}
I have tried 2 different amounts of hidden layer neurons, 90 and 110:
\includenamedimage[width=0.6\textwidth]{h90.png}{90 Hidden Layer Neurons}
\begin{table}[h!]
  \centering
  \begin{tabular}{|c|c|}
  \hline
  \textbf{Train Error} & \textbf{Test Error} \\
  \hline
  0.0731 & 0.1199 \\
  \hline
  \end{tabular}
  \caption*{Reference Solution Error Rates}
  \end{table}
\includenamedimage[width=0.6\textwidth]{h110.png}{110 Hidden Layer Neurons}
\begin{table}[h!]
  \centering
  \begin{tabular}{|c|c|}
  \hline
  \textbf{Train Error} & \textbf{Test Error} \\
  \hline
  0.0745 & 0.1213 \\
  \hline
  \end{tabular}
  \caption*{Reference Solution Error Rates}
  \end{table}

The results were underwhelming, as the reference solution still had the lowest error in the test set, the one that matters.
\subsection{Learning Rate}
The next hyperparameter I tested was the learning rate, with 2 different values, 0.005 and 0.0005:
\includenamedimage[width=0.6\textwidth]{lr0005.png}{Learning Rate 0.005}
\begin{table}[h!]
  \centering
  \begin{tabular}{|c|c|}
  \hline
  \textbf{Train Error} & \textbf{Test Error} \\
  \hline
  0.0810 & 0.1275 \\
  \hline
  \end{tabular}
  \caption*{Reference Solution Error Rates}
  \end{table}
\includenamedimage[width=0.6\textwidth]{lr00005.png}{Learning Rate 0.0005}
\begin{table}[h!]
  \centering
  \begin{tabular}{|c|c|}
  \hline
  \textbf{Train Error} & \textbf{Test Error} \\
  \hline
  0.0828 & 0.1217 \\
  \hline
  \end{tabular}
  \caption*{Reference Solution Error Rates}
  \end{table}

Once again, these results were outperformed by the reference solution.
Therefore, for trying new possible improvements, I will use the reference solution as the base.

\section{Possible Improvements}
\subsection{Data Normalization}
The first possible improvement tried was normalization (standardization) of the data around the mean.
\includenamedimage[width=0.6\textwidth]{norm.png}{Normalized Data Solution}
\begin{table}[h!]
  \centering
  \begin{tabular}{|c|c|}
  \hline
  \textbf{Train Error} & \textbf{Test Error} \\
  \hline
  0.0332 & 0.122 \\
  \hline
  \end{tabular}
  \caption*{Reference Solution Error Rates}
  \end{table}
There were big improvements in the error rates for the training set (From ~0.07 to ~0.03), but no improvements in the test set, meaning possible overfitting.
This means the hidden layer neuron size can be decreased.
\subsection{Momentum}
As the next possible improvement, momentum was added to the learning process:
\includenamedimage[width=0.6\textwidth]{moment.png}{Momentum Solution}
\begin{table}[h!]
  \centering
  \begin{tabular}{|c|c|}
  \hline
  \textbf{Train Error} & \textbf{Test Error} \\
  \hline
  0.0739 & 0.1227 \\
  \hline
  \end{tabular}
  \caption*{Reference Solution Error Rates}
  \end{table}
While it took a bit more time to train each epoch (roughly 3.5 seconds more), the improvement was faster in the beginning compared to the reference, then it slowed down and once again the reference solution outperformed it.
\subsection{Combined Solution}
Observing that both the normalization and momentum solutions still had similar error rates in the test set, but took less epochs to reach them, both were combined and the training was done for 20 epochs:
\includenamedimage[width=0.6\textwidth]{final.png}{Momentum and Normalized Data Solution}
\begin{table}[h!]
  \centering
  \begin{tabular}{|c|c|}
  \hline
  \textbf{Train Error} & \textbf{Test Error} \\
  \hline
  0.0637 & 0.127 \\
  \hline
  \end{tabular}
  \caption*{Reference Solution Error Rates}
  \end{table}
\section{Final Comparison}
Overall, I could not achieve any significant improvements in lowering the error rate for the test set.
I did however managed to decrease the hidden layer neuron needed (100 to 90) as well as the epoch count (50 to 20), while having relatively similar results.
Below is the error rate comparison for the reference and the final solution:
\begin{table}[h!]
  \centering
  \begin{tabular}{|c|c|c|}
  \hline
  \textbf{Model} & \textbf{Train Error} & \textbf{Test Error} \\
  \hline
  Reference & 0.0721 & 0.1194 \\
  \hline
  Final & 0.0637 & 0.127 \\
  \hline
  \end{tabular}
  \caption*{Reference Solution Error Rates}
  \end{table}

Below are the confusion matrices for the reference and the final solution:
\begin{table}[h!]
  \centering
  \begin{tabular}{c|*{10}{c}|c}
    & \multicolumn{10}{c|}{Predicted} & \\
  \hline
  Actual & 0 & 1 & 2 & 3 & 4 & 5 & 6 & 7 & 8 & 9 & Total \\
  \hline
  0 & 861 & 2 & 20 & 25 & 7 & 0 & 71 & 0 & 14 & 0 & 1000 \\
  1 & 3 & 967 & 3 & 19 & 5 & 0 & 1 & 0 & 2 & 0 & 1000 \\
  2 & 18 & 0 & 776 & 13 & 126 & 0 & 61 & 0 & 6 & 0 & 1000 \\
  3 & 26 & 13 & 11 & 891 & 35 & 0 & 18 & 0 & 6 & 0 & 1000 \\
  4 & 0 & 0 & 56 & 35 & 857 & 0 & 44 & 0 & 8 & 0 & 1000 \\
  5 & 0 & 0 & 1 & 1 & 0 & 944 & 0 & 33 & 2 & 19 & 1000 \\
  6 & 143 & 0 & 76 & 38 & 94 & 0 & 627 & 0 & 22 & 0 & 1000 \\
  7 & 0 & 0 & 0 & 0 & 0 & 18 & 0 & 961 & 0 & 21 & 1000 \\
  8 & 2 & 0 & 3 & 8 & 5 & 1 & 5 & 6 & 970 & 0 & 1000 \\
  9 & 0 & 0 & 0 & 0 & 0 & 6 & 0 & 41 & 1 & 952 & 1000 \\
  \hline
  \end{tabular}
  \caption*{Reference Confusion Matrix}
  \end{table}
  
  % Final Confusion Matrix
  \begin{table}[h!]
  \centering
  \begin{tabular}{c|*{10}{c}|c}
    & \multicolumn{10}{c|}{Predicted} & \\
  \hline
  Actual & 0 & 1 & 2 & 3 & 4 & 5 & 6 & 7 & 8 & 9 & Total \\
  \hline
  0 & 876 & 3 & 17 & 20 & 4 & 2 & 65 & 3 & 10 & 0 & 1000 \\
  1 & 8 & 965 & 0 & 17 & 4 & 0 & 6 & 0 & 0 & 0 & 1000 \\
  2 & 24 & 0 & 770 & 9 & 120 & 1 & 71 & 1 & 4 & 0 & 1000 \\
  3 & 40 & 11 & 14 & 875 & 36 & 2 & 19 & 0 & 3 & 0 & 1000 \\
  4 & 2 & 3 & 82 & 29 & 828 & 0 & 51 & 1 & 4 & 0 & 1000 \\
  5 & 0 & 0 & 1 & 1 & 0 & 931 & 0 & 44 & 2 & 21 & 1000 \\
  6 & 181 & 1 & 91 & 29 & 78 & 2 & 605 & 1 & 12 & 0 & 1000 \\
  7 & 0 & 0 & 0 & 0 & 0 & 9 & 0 & 970 & 0 & 21 & 1000 \\
  8 & 2 & 0 & 7 & 7 & 3 & 1 & 11 & 6 & 963 & 0 & 1000 \\
  9 & 0 & 0 & 0 & 2 & 0 & 8 & 1 & 42 & 0 & 947 & 1000 \\
  \hline
  \end{tabular}
  \caption*{Final Confusion Matrix}
  \end{table}
\end{document}